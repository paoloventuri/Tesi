\documentclass[12pt]{report}
\renewcommand{\baselinestretch}{1.3}      % interline spacing
%
% \includeonly{}
%
%			PREAMBOLO
%
\usepackage[a4paper]{geometry}
\usepackage{amssymb,amsmath,amsthm}
\usepackage{graphicx}
\usepackage{url}
\usepackage{hyperref}
\usepackage{epsfig}
\usepackage[italian]{babel}
\usepackage{tesi}

% per le accentate
\usepackage[utf8]{inputenc}
%
\newtheorem{myteor}{Teorema}[section]
%
\newenvironment{teor}{\begin{myteor}\sl}{\end{myteor}}
%
%
%			TITOLO
%
\begin{document}
\title{Spazi-Unimi: \\Progettazione e implementazione dell'integrazione e validazione delle diverse fonti di dati edilizi}
\author{Paolo Venturi}
\dept{Corso di Laurea in Informatica} 
\anno{2013-2014}
\matricola{775021}
\relatore{Prof. Carlo Bellettini}
\correlatore{Dr. Matteo Camilli}
%
%        \submitdate{month year in which submitted to GPO}
%		- date LaTeX'd if omitted
%	\copyrightyear{year degree conferred (next year if submitted in Dec.)}
%		- year LaTeX'd (or next year, in December) if omitted
%	\copyrighttrue or \copyrightfalse
%		- produce or don't produce a copyright page (false by default)
%	\figurespagetrue or \figurespagefalse
%		- produce or don't produce a List of Figures page
%		  (false by default)
%	\tablespagetrue or \tablespagefalse
%		- produce or don't produce a List of Tables page
%		  (false by default)
% 
%			DEDICA
%
\beforepreface
\prefacesection{}
        {\hfill \Large {\sl dedicato a \dots}}
% 
%			PREFAZIONE
%
\prefacesection{Prefazione}
hkjafgyruet.
%
%
%			ORGANIZZAZIONE
\section*{Organizzazione della tesi}
\label{organizzazione}
La tesi \`e organizzata come segue:
\begin{itemize}
\item nel Capitolo 1 ....
\end{itemize}
%
%			RINGRAZIAMENTI
%
\prefacesection{Ringraziamenti}
asdjhgftry.
\afterpreface
% 
% 
%			CAPITOLO 1
\chapter{Introduzione al progetto Spazi-Unimi}
\label{cap1}
Il progetto Spazi-Unimi nasce dall’esigenza degli utenti (studenti, professori, etc.) dell’Università degli Studi di Milano di cercare in modo facile e veloce la posizione delle aule di loro interesse. 
Vista la dislocazione delle sedi universitarie in varie aree della città (e della regione) un nuovo studente o un visitatore può avere serie difficoltà nell’orientarsi: da qui l’idea di creare una App che semplifichi la ricerca degli edifici universitari e delle loro stanze. 

Spazi-Unimi è stato ideato nell’ambito del progetto Campus Sostenibile, una collaborazione tra il Politecnico di Milano e l’Università degli Studi di Milano, che si propone di trasformare il quartiere Città Studi in un modello per quanto riguarda la qualità della vita e la sostenibilità.
Dalla proposta del Prof. Carlo Bellettini è quindi partito lo sviluppo del progetto che è stato portato avanti con altri due studenti del dipartimento di Informatica: Samuel Brandao Gomes e Diego Costantino.

I file da cui estrarre le informazioni utili alla creazione dell’applicazione sono stati forniti da due diverse fonti:
\begin{itemize}
\item la Divisione Manutenzione edilizia e impiantistica che ha concesso le piantine delle sedi universitarie e le informazioni sulle aule didattiche;
\item la Divisione sistemi informativi che ha concesso le informazioni sulle aule presenti sul sistema EasyRoom.
\end{itemize}

Durante le 18 settimane di stage interno il lavoro effettuato ha riguardato principalmente lo sviluppo della parte back-end che si propone di fornire agli addetti delle diverse fonti di dati un modo semplice e immediato per aggiornare le informazioni.
La parte su cui più si è incentrato il mio lavoro è stata l'unione dei dati provenienti dalle diverse fonti cercando di rendere disponibili all'utente finale le migliori informazioni per quanto riguarda completezza e qualità.

Nell'untima parte dello stage invece ci si è concentrati sulla definizione di un'interfaccia REST API utile alle nostre necessità e all'eleborazione di una semplice applicazione (ANDROID/WEB?) come prototipo di dimostrazione per il futuro sviluppo di un'applicazione multipiattaforma utile agli utenti dell'università.
     
%
%

%
%			BIBLIOGRAFIA
%
\begin{thebibliography}{00}
%
\bibitem{gotti91}
M. Gotti, I linguaggi specialistici, Firenze, La Nuova Italia, 1991.
%
\bibitem{wellek62}
R. Wellek, A. Warren, Theory of Literature , 3rd edition, New York, Harcourt, 1962.
%
\bibitem{canziani78}
A. Canziani et al., Come comunica il teatro: dal testo alla scena. Milano, Il Formichiere, 1978.
%
\bibitem{MoD67}
Ministry of Defence, Great Britain, Author and Subject Catalogues of the Naval Library, London, Ministry of Defence, HMSO, 1967.
%
\bibitem{heine23}
H. Heine, Pensieri e ghiribizzi. A cura di A. Meozzi. Lanciano, Carabba, 1923.
%
\bibitem{basso62}
L. Basso, ``Capitalismo monopolistico e strategia operaia'', Problemi del socialismo, vol. 8, n. 5, pp. 585-612, 1962.
%
\bibitem{avirovic93}
L. Avirovic, J. Dodds (a cura di), Atti del Convegno internazionale "Umberto Eco, Claudio Magris. Autori e traduttori a confronto" ( Trieste, 27-28 novembre 1989), Udine, Campanotto, 1993.
%
\bibitem{gans67}
E.L. Gans, "The Discovery of Illusion: Flaubert's Early Works, 1835-1837", unpublished Ph.D. Dissertation, Johns Hopkins University, 1967.
%
\bibitem{harrison92}
R. Harrison, Bibliography of planned languages (excluding Esperanto).  \url{http://www.vor.nu/langlab/bibliog.html}, 1992, agg. 1997.
%
\end{thebibliography}
% 
\end{document}


 
